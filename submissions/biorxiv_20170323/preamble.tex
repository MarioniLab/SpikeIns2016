\documentclass[10pt,letterpaper]{article}
\usepackage[top=0.85in,left=2.75in,footskip=0.75in,marginparwidth=2in]{geometry}

% use Unicode characters - try changing the option if you run into troubles with special characters (e.g. umlauts)
\usepackage[utf8]{inputenc}

% clean citations
\usepackage{cite}

% hyperref makes references clicky. use \url{www.example.com} or \href{www.example.com}{description} to add a clicky url
\usepackage{nameref,hyperref}

% line numbers
\usepackage[right]{lineno}

% improves typesetting in LaTeX
\usepackage{microtype}
\DisableLigatures[f]{encoding = *, family = * }

% text layout - change as needed
\raggedright
\setlength{\parindent}{0.5cm}
\textwidth 5.25in 
\textheight 8.75in

% Remove % for double line spacing
%\usepackage{setspace} 
%\doublespacing

% use adjustwidth environment to exceed text width (see examples in text)
\usepackage{changepage}

% adjust caption style
\usepackage[aboveskip=1pt,labelfont=bf,labelsep=period,singlelinecheck=off]{caption}

% remove brackets from references
\makeatletter
\renewcommand{\@biblabel}[1]{\quad#1.}
\makeatother

% headrule, footrule and page numbers
\usepackage{lastpage,fancyhdr,graphicx}
\usepackage{epstopdf}
\pagestyle{myheadings}
\pagestyle{fancy}
\fancyhf{}
\rfoot{\thepage/\pageref{LastPage}}
\renewcommand{\footrule}{\hrule height 2pt \vspace{2mm}}
\fancyheadoffset[L]{2.25in}
\fancyfootoffset[L]{2.25in}

% use \textcolor{color}{text} for colored text (e.g. highlight to-do areas)
\usepackage{color}

% define custom colors (this one is for figure captions)
\definecolor{Gray}{gray}{.25}

% this is required to include graphics
\usepackage{graphicx}

% use if you want to put caption to the side of the figure - see example in text
\usepackage{sidecap}

% use for have text wrap around figures
\usepackage{wrapfig}
\usepackage[pscoord]{eso-pic}
\usepackage[fulladjust]{marginnote}
\reversemarginpar

% Adding multirow.
\usepackage{multirow}

% Other required things:
\usepackage{textgreek}
\usepackage{color}

\newcommand{\suppfignorm}{1}
\newcommand{\suppfigtotals}{2}
\newcommand{\suppfigorder}{3}
\newcommand{\suppfigbiophys}{4}
\newcommand{\suppfignoise}{5}

\newcommand{\suppsecmath}{1}
\newcommand{\suppsecnoise}{2}
\newcommand{\suppsecsim}{3}

\newcommand{\supptabstats}{1}

% document begins here
\begin{document}
\vspace*{0.35in}

% title goes here:
\begin{flushleft}
{\Large
    \textbf\newline{Assessing the reliability of spike-in normalization for analyses of single-cell RNA sequencing data}
}
\newline
% authors go here:
\\
Aaron T. L. Lun\textsuperscript{1}, 
Fernando J. Calero-Nieto\textsuperscript{2},
Liora Haim-Vilmovsky\textsuperscript{3,4}, 
Berthold G\"ottgens\textsuperscript{2},
John C. Marioni\textsuperscript{1,3,4,*}
\\
\bigskip
\bf{1} Cancer Research UK Cambridge Institute, University of Cambridge, Li Ka Shing Centre, Robinson Way, Cambridge CB2 0RE, United Kingdom
\\
\bf{2} Wellcome Trust and MRC Cambridge Stem Cell Institute, University of Cambridge, Wellcome Trust/MRC Building, Hills Road, Cambridge CB2 0XY, United Kingdom
\\
\bf{3} EMBL European Bioinformatics Institute, Wellcome Genome Campus, Hinxton, Cambridge CB10 1SD, United Kingdom
\\
\bf{4} Wellcome Trust Sanger Institute, Wellcome Genome Campus, Hinxton, Cambridge CB10 1SA, United Kingdom
\\
\bigskip
* marioni@ebi.ac.uk

\end{flushleft}

\newcommand{\revised}[1]{#1}

\section*{Abstract}
By profiling the transcriptomes of individual cells, single-cell RNA sequencing provides unparalleled resolution to study cellular heterogeneity.
However, this comes at the cost of high technical noise, including cell-specific biases in capture efficiency and library generation.
One strategy for removing these biases is to add a constant amount of spike-in RNA to each cell, and to scale the observed expression values so that the coverage of spike-in RNA is constant across cells.
This approach has previously been criticized as its accuracy depends on the precise addition of spike-in RNA to each sample, and on similarities in behaviour (e.g., capture efficiency) between the spike-in and endogenous transcripts.
Here, we perform mixture experiments using two different sets of spike-in RNA to quantify the variance in the amount of spike-in RNA added to each well in a plate-based protocol.
We also obtain an upper bound on the variance due to differences in behaviour between the two spike-in sets.
We demonstrate that both factors are small contributors to the total technical variance and have only minor effects on downstream analyses such as detection of highly variable genes and clustering.
Our results suggest that spike-in normalization is reliable enough for routine use in single-cell RNA sequencing data analyses.

% now start line numbers
\linenumbers

