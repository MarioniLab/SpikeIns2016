\documentclass{article}
\usepackage{graphicx}
\usepackage[margin=2.5cm]{geometry}
\usepackage[labelfont=bf]{caption}
\usepackage{color}
\usepackage{xcite}
\usepackage{amsmath}
\usepackage{textcomp}
\usepackage{natbib}

\renewcommand{\textfraction}{1.0}
\renewcommand{\floatpagefraction}{.9}
\newcommand\revised[1]{\textcolor{red}{#1}}
\renewcommand{\topfraction}{0.9}    % max fraction of floats at top
\renewcommand{\bottomfraction}{0.8} % max fraction of floats at bottom
\renewcommand{\textfraction}{0.07}  % allow minimal text w. figs

\makeatletter 
\renewcommand{\thefigure}{S\@arabic\c@figure} 
\renewcommand{\thetable}{S\@arabic\c@table} 

\usepackage{url}
\urlstyle{same}

\begin{document}

\begin{titlepage}
\vspace*{3cm}
\begin{center}

{\LARGE
When normalization gets prickly: are spike-ins good enough for single-cell RNA sequencing data?
\par}

\vspace{0.75cm}

{\Large 
    \textsc{Supplementary Materials}
\par
}
\vspace{0.75cm}

\large
by


\vspace{0.75cm}
Aaron T. L. Lun$^{1}$ and John C. Marioni$^{1,2}$

\vspace{1cm}
\begin{minipage}{0.9\textwidth}
\begin{flushleft} 
$^1$Cancer Research UK Cambridge Institute, University of Cambridge, Li Ka Shing Centre, Robinson Way, Cambridge CB2 0RE, United Kingdom \\[6pt]
$^2$EMBL European Bioinformatics Institute, Wellcome Genome Campus, Hinxton, Cambridge CB10 1SD, United Kingdom \\[6pt]
\end{flushleft}
\end{minipage}

\vspace{1.5cm}
{\large \today{}}

\vspace*{\fill}
\end{center}
\end{titlepage}

\section{Decomposing the variance for log-ratios of spike-ins}

\subsection{Overview of basic definitions}
The first step of the protocol involves adding spike-in populations together.
Consider the following definitions:
\begin{center}
    \begin{tabular}{l p{0.9\textwidth}}
$C_{t_s}$  & The concentration in molecules per unit of volume for transcript $t_s$ in spike-in population $s$. \\
$\eta_{is}$   & The volume of spike-in population $s$ added to well $i$. \\
$\zeta_{it_s}$ & The number of molecules for transcript $t_s$ of population $s$ added to $i$ during spike-in addition.
\end{tabular}
\end{center}
$\eta_{is}$ is a random variable that depends on the experimental precision of volume addition.
In contrast, $C_{t_s}$ is a constant that depends on the composition of the spike-in population $s$.
We also define $\zeta_{it_s} = C_{t_s}\eta_{is}$.

After the spike-in volumes are added together, library preparation is performed to convert RNA to cDNA prior to high-throughput sequencing.
Consider the following definitions:
\begin{center}
\begin{tabular}{l p{0.9\textwidth}}
$R_{t_s}$ & The optimal number of cDNA fragments generated from each molecule of $t_s$. \\
$\psi_{i}$ & The overall efficiency of cDNA generation in well $i$ for all transcripts. \\
$\phi_{is}$ & The relative efficiency of cDNA generation in well $i$ for all transcripts of population $s$. \\
$\delta_{it_s}$ & The cDNA fragment-per-transcript molecule conversion rate for transcript $t_s$ in well $i$.
\end{tabular}
\end{center}
$R_{t_s}$ is a constant as it depends on the inherent properties of each transcript.
$\psi_{i}$ is a random variable that represents well-specific biases in library preparation.
Additional biases in each well are modelled by $\phi_{is}$, which represents well- and population-specific behaviour.
These parameters are related by $\delta_{it_s} = R_{t_s} \phi_{i} \psi_{is}$.

Finally, define $\nu_{ig}$ as the number of transcript molecules for gene $g$ in the cellular RNA for the cell in well $i$.
This is a random variable that includes the biological variability in gene expression.
Let $\delta_{ig}$ represent the cDNA fragment-per-transcript conversion rate for $g$ in $i$, which is also a random variable across wells.

All random variables are assumed to be independent unless they have been explicitly related by equalities, e.g., $\zeta_{it_s}$ and $\eta_{is}$.
This is justified by the fact that they describe separate processes in the protocol.

\subsection{Constructing the distribution of the total counts}
For spike-in populations $s=1, 2, \ldots$, the cDNA fragment-to-read conversion factor for each well is
\[
    \rho_i = \frac{N_i}{\sum_g \delta_{ig}\nu_{ig} + \sum_{t_1} \delta_{it_1} S_{it_1}  + \sum_{t_2} \delta_{it_2} S_{it_2} + \ldots} 
\]
where $N_i$ is the size of the library for well $i$.
This accounts for composition biases, e.g., when overexpression of one gene can take up more sequencing resources and suppress the coverage of other genes.
In contrast, for UMI data, the fragment-to-UMI conversion factor is simply $\rho_i = 1$ when sequencing is saturated.

The conditional expectation of the total read count across all transcripts in population $s$ for well $i$ is
\[
    \mu_{is} = \rho_i  \textstyle\sum_{t_s} \delta_{it_s} \zeta_{it_s} = \rho_i  \eta_{is} \psi_i \phi_{is} \sum_{t_s} R_{t_s} C_{t_s} \;.
\]
The observed total $y_{is}$ is assumed to be log-normally distributed, conditional on the other parameters, i.e., 
\[
    \log y_{is} | \{ \eta_{is}, \phi_{is} \forall s \}, \psi_i, \{ \delta_{ig}, \nu_{ig} \forall g \} \sim \mathcal{N}(\log \mu_{is}, \sigma^2_{lib(s)})
\]
where $\sigma^2_{lib(s)}$ represents the variability due to sequencing of population $s$.
We consider that $\sigma^2_{lib(s)} = f(A_s)$ for some monotonic decreasing function $f$ of the abundance $A_s$ for $s$.
This represents the decrease in variability with increasing abundance, as typically observed in sequencing data.
For simplicity, we define $A_s = E(\log \mu_{is})$ rather than $A_s = \log \mu_{is}$, i.e., we use the full expectation without conditioning on any other parameters.
This approximation ensures that we do not have to consider the effect of variability in $V_{is}$, $\psi_i$, etc. on the value of $\sigma^2_{lib(s)}$.
We also assume that totals are conditionally independent between spike-in populations.
This is because we have already conditioned on the well-specific biases common to the populations within each well.
Thus, any remaining variation due to technical noise should be independent for each population $s$.

\subsection{Deriving the variance of the log-ratios upon separate addition}
Denote the log-ratio of the total counts between two spike-in populations $s=1, 2$ as $\theta_i = \log(y_{i1}/y_{i2})$ for well $i$.
Based on the distribution of the total counts, $\theta_i$ is conditionally distributed as
\[
    \theta_i |  \{ \eta_{is}, \phi_{is} \forall s \} \sim \mathcal{N}( \log \mu_{i1} - \log \mu_{i2}, \sigma^2_{lib(1)} + \sigma^2_{lib(2)} ) \;. 
\]
There is no dependence on $\delta_{ig}$, $\nu_{ig}$ or $\psi_i$, as these terms cancel out during calculation of $\theta_i$ for each well.

Further assume that both $\log V_{ix}$ and $\log V_{iy}$ are normally distributed with variance $\sigma^2_{vol}$.
The use of the same variance is justified by the fact that the same spike-in volume is added for both $x$ and $y$, such that the variability should be the same for both volumes.
The assumption means that the conditional mean $E(\theta_i | \{ V_{is}, \psi_{is} \forall s \})$ is normally distributed, such that $\theta_i$ itself is also normally distributed with variance
\[
    \mbox{var}(\theta_i) = 2\sigma_{vol}^2 + \sigma^2_{lib(1)} + \sigma^2_{lib(2)} + \mbox{var}(\log \phi_{i1}) +  \mbox{var}(\log \phi_{i2}) \;. 
% Easily derived using the law of total variances.
% Normal distribution for R_i is obvious, based on product of PDFs (int[ P(X=x|u)*P(u) du ] gives a exp(x^2), eventually, and you can work this out to validate the variance).
% Both claims tested in simulations (sample variances of normal distribution with normal mean match up with expected truth, no rejections on a Shapiro-Wilk test).
\]
This can be estimated as the sample variance of the log-ratios across all wells in the first experiment. 

\subsection{Deriving the variance of the log-ratios with premixing}
In the second experiment, mixing of spike-in populations occurs beforehand such that addition is not variable.
More specifically, $\eta_{i1}=\eta_{i2}$ in each well such that they cancel out in calculating $\theta_i$ and do not contribute to the variance.
This means that the variance of the log-ratio (denoted here as $\theta_i'$) can be written as
\[
    \mbox{var}(\theta_i') = \sigma^2_{lib(1)} + \sigma^2_{lib(2)} + \mbox{var}(\log \phi_{i1}) +  \mbox{var}(\log \phi_{i2})
\]
and can be estimated as the sample variance of the log-ratios in this experiment.
Subtraction of the estimated variance of $\theta_i'$ from that of $\theta_i$ from the first experiment can be used to obtain an estimate of $\sigma^2_{vol}$.
Note that this assumes that $E(\log \mu_{is})$ is the same for each spike-in population between the first and second experiments, i.e., the average count size for each population is the same between experiments.
This ensures that there are no abundance-related changes in the technical variability of each population.
In general, this assumption is reasonable if premixing of the two populations is done in a 1:1 ratio (volume-by-volume) and the total added volume of premixed spike-in is equal to the total volume for separate additions in the first experiment.

% It's possible to do something more complicated with a Poisson sampling distribution that's conditional on the mean.
% You could use the general variance decomposition formula to get the variance of the total count (see Bowsher and Swain, 2012).
% This would account for the mean-variance relationship at low counts.
% However, it becomes intractable to solve for the variance of the log-ratio of counts.
% There is no apparent benefit to this complexity, given that the log-normal approximation is more than sufficient at large counts.

\subsection{Estimating the variance of the population-specific capture efficiency}
Some work is required to decompose the components of $\mbox{var}(\theta_i')$, in order to separate technical variability from the variability of the $\phi_{is}$ terms.
This can be done by splitting one of the spike-in populations into two halves.
Divide transcripts in $s=1$ into two halves, i.e., $s=1_a$ and $1_b$.
As $\phi_{i1} = \phi_{i1_a} = \phi_{i1_b}$ for transcripts from the same spike-in population, they cancel out when computing the log-ratio between the total counts of the two halves.
Thus, the variance of the log-ratio (denoted here as $\theta_{i(ab)}'$) will simplify to 
\[
    \mbox{var}(\theta_{i(ab)}') = \sigma^2_{lib(1_a)} + \sigma^2_{lib(1_b)} \;.
\]
This is true for both experiments, as $\eta_{i1} = \eta_{i1_a} = \eta_{i1_b}$ within a single population.
Now, consider the variance of the log-ratio of the total count for each half to the second population $s=2$.
This is written as
\begin{align*}
    \mbox{var}(\theta_{i(a)}') &= \sigma^2_{lib(1_a)} + \sigma^2_{lib(2)} + \mbox{var}(\log \phi_{i1}) + \mbox{var}(\log \phi_{i2}) \\
    \mbox{var}(\theta_{i(b)}') &= \sigma^2_{lib(1_b)} + \sigma^2_{lib(2)} + \mbox{var}(\log \phi_{i1}) + \mbox{var}(\log \phi_{i2})
\end{align*}
using only the samples in the second experiment (where variability in volume addition need not be considered).
We now perform some basic arithmetic on the variances of the log-ratios:
\begin{align*}
    &\mbox{var}(\theta_{i(a)}') +  \mbox{var}(\theta_{i(b)}') - \mbox{var}(\theta_{i(ab)}') = 2 [\sigma^2_{lib(2)} + \mbox{var}(\log \phi_{i1}) + \mbox{var}(\log \phi_{i2}) ] \\
    &\mbox{var}(\theta_i') - \textstyle\frac{1}{2}[\mbox{var}(\theta_{i(a)}') + \mbox{var}(\theta_{i(b)}') - \mbox{var}(\theta_{i(ab)}') ] = \sigma^2_{lib(1)}
\end{align*}
We repeat this after splitting $s=2$ into halves to obtain an estimate of $\sigma^2_{lib(2)}$.
These values can be subtracted from $\mbox{var}(\theta_i')$ to estimate the variance in spike-in-specific behaviour, i.e., $\mbox{var}(\log \phi_{i1}) +  \mbox{var}(\log \phi_{i2})$.

Note that our model does not consider transcript-specific variability in behaviour in each well.
If such variability exists, it will not cancel out when calculating $\theta_{i(ab)}')$ and will instead be treated as part of the technical variability.
This results in overestimation of $\sigma^2_{lib(s)}$ and underestimation of the variance in spike-in-specific behaviour.
To mitigate this effect, we split the transcripts so that the distribution of transcript abundances is similar between halves.
This increases the chance that each half will behave similarly, such that spike-in-specific behaviour will cancel out between halves in each well when calculating the log-ratio.

\newpage
\section{Dealing with variable spike-in behaviour}
Another implicit assumption is that the spike-in populations exhibit equivalent behaviour with respect to cDNA generation efficiency.
This allows $\psi_i$ to cancel out during calculation of $R_i$, such that any variability in efficiency across wells can be ignored.
However, this may not be true for populations like the set of ERCC spike-ins with, e.g., differing GC composition, shorter poly-A tails.
Differences in well-specific efficiency can be modelled by introducing a $\psi_{is}$ term to replace $\psi_i$ in the above calculations.
This means that the conditional expectation of $R_i$ with respect to the random variables $V_{i}$ and $\psi_{is}$ becomes
\[
E(R_i | V_{ix}, V_{iy},  \psi_{ix}, \psi_{iy} ) = \log \frac{ S_{ix} \sum_t \psi_{ix}\delta_{t} \pi_{tx} } { S_{iy}\sum_t \psi_{iy}\delta_{t} \pi_{ty} } \;. \\
\]
The effect on the variance of $R_i$ depends on whether the ratio $\psi_{ix}/\psi_{iy}$ is constant or variable.
If it is constant, it will be absorbed into the constant term $Z$ and will not contribute to the variance across wells.
Otherwise, we can model $\log(\psi_{ix}/\psi_{iy})$ with a normal distribution that is independent of $V_{is}$.
The variance of this distribution will contribute directly to both $\mbox{var}(R_i)$ and $\mbox{var}(R_i')$, and will be effectively calculated as part of $\sigma^2_{lib}$.
Subtraction of sample variances will subsequently yield an estimate of $\sigma^2_{vol}$, as before.

Disentangling the variability of $\log(\psi_{ix}/\psi_{iy})$ from the other components of $\sigma^2_{lib}$ is more difficult.
It requires careful control over the properties of the spike-in populations -- namely, by using two populations that are effectively identical for the purposes of reverse transcription, yet sufficiently different for read alignment to the correct genome.
Some study of the relative contribution of this variance term can be performed by using several different spike-ins, and observing differences in the estimated $\sigma^2_{lib}$ (assuming that variance in the other steps of the protocol are constant across runs).
If the variance is negligible, then the exact value of $\psi_{ix}/\psi_{iy}$ is irrelevant in most applications as it will have no effect on the normalization factors between cells.

Another approach is to shuffle the spike-in sets \textit{in silico} for the second experiment.
Half the species in one set are summed with half the species in the other set, and this is repeated with the remaining species in both sets.
The variance of the log-ratios between these two sums is computed across wells, using the same partitioning of species for all wells.
To ensure robustness, a different partition can be chosen and the variance calculation repeated; the average variance can be computed across partitions.
The idea is that, on average, the shuffled populations do not exhibit any systematic difference in their capture efficiency.
Any variability in the log-ratio is attributable to the protocol alone, rather than changes to efficiency.
Subtracting from the original variance of the control experiment will identify the variability due to spike-in behaviour.

\newpage
\section{Statistical analyses of the results}
Denote the sample variance of $R$ as $s^2_1$, and that of $R'$ as $s^2_2$.
These are estimated from the set of observed ratios in each experiment, using the corrected Pearson estimator.
We could then intuitively define
\begin{align*}
\widehat{\sigma^2_{lib}} &= s^2_2 \quad\mbox{and} \\
\widehat{\sigma^2_{vol}} &= \frac{s^2_1 - s^2_2}{2} \;.
\end{align*}
However, this is not sensible if $s^2_1 < s^2_2$ as the estimate for $\sigma^2_{vol}$ will be negative.
In such cases, 
\begin{align*}
\widehat{\sigma^2_{vol}} &= 0 \quad\mbox{and} \\
\widehat{\sigma^2_{lib}} &= \frac{s^2_1(n_1 - 1) + s^2_2(n_2-1)}{n_1 + n_2 - 2} 
\end{align*}
where $n_1$ and $n_2$ are the number of repeated wells in the first and second experiments, respectively.
This uses all of the data from the first and second experiments to estimate $\sigma^2_{lib}$, while the estimate of $\sigma^2_{vol}$ is fixed at the lower bound of zero.
These definitions are based on the principle of restricted maximum likelihood when the parameter space is constrained to non-negative values (see Thompson, 1962).

% See "The Problem of Negative Estimates of Variance Components". Thompson (1962), Ann. Math. Statist., Volume 33, Number 1 (1962), 273-289.
% More specifically, the log-REML is equal to:
%\[
%L(\mathbf{y_1}, \mathbf{y_2} | \sigma^2_{lib}, \sigma^2_{vol}) = - n_1 \log(\sigma^2_{lib}) - \frac{1}{\sigma^2_{lib}} \sum y_{1i} - n_2 \log(\sigma^2_{lib} + \sigma^2_{vol}) - \frac{1}{\sigma^2_{lib} + \sigma^2_{vol}} \sum y_{2i} 
%\]
% where y_1 and y_2 are vectors of residual effects from the two experiments.
% If you take the partial derivatives with respect to each sigma^2, you get two equations that can be solved simultaneously.
% This yields a single maximum for the REML at the standard expressions for the experimental sample variances (and after subtraction, for the volume variance).
% However, the story's different when the volume variance estimate is forced to zero.
% Under normal conditions, there's only one maxima (see above) - but in this case, the maxima lies outside the constrained space.
% This means that the maxima in the constrained space must lie on the boundary, as the REML must increase as it gets closer to the maxima.
% We can then plug in a zero value for the volume variance and solve for the library variance, which gives us the value above. 

It is also possible to determine the significance of non-zero estimates for $\sigma^2_{vol}$.
This aims to determine whether there is a significant increase in variability from volume addition, over the variability introduced during general library preparation.
Specifically, the null hypothesis states that $\sigma^2_{vol} = 0$.
This means that
\begin{align*}
s^2_1 &\sim \frac{\sigma^2_{lib} \chi^2_{n_1 - 1}}{n_1 - 1} \quad\mbox{and} \\
s^2_2 &\sim \frac{\sigma^2_{lib} \chi^2_{n_2 - 1}}{n_2 - 1} 
\end{align*}
As the sample variances are estimated independently, the ratio $s^2_1/s^2_2$ follows a F-distribution on $n_1-1$ and $n_2-1$ degrees of freedom under the null.
The observed value of this ratio can be used to compute a $p$-value based on the upper tail of this distribution.
A similar test can be used to determine whether alterations in the protocol result in significant decreases in the variance, i.e., improvements in spike-in reliability.

Of course, these analyses hinge on the assumption of normality for the log-ratios. 
This can be evaluated empirically using established statistical methods such as the Shapiro-Wilk test.
Simulations indicate that the log-ratios of overdispersed count data can be accurately modelled with the normal distribution.

An alternative approach to variance estimation is to take the median absolute deviation (MAD) for each set of ratios and multiply it by 1.4826.
This will yield a robust, unbiased estimate of the standard deviation under normality.
While this protects against outliers, it will likely invalidate the tests described above -- the variance of this estimator will be different from the Pearson estimator.
Rather, it may be possible to use simulations to obtain a null distribution for the variance ratios.
For example, you could simulate the ratios of MAD-based variance estimates for the standard normal distribution.
The final distribution of these ratios should be the same even if you change the location or scale of the original normal distribution.

\newpage
\section{Estimating the variability of total cellular RNA}
Our aim is to estimate the variance of total cellular RNA, and to compare it to the variance of the spike-ins.
If the former is negligible compared to the latter, we can claim that any imprecision in spike-in addition will have little impact on the biological conclusions.
This suggests that spike-ins can be safely used.

To test this, we would ideally estimate $\mbox{var}(\sum_t \nu_{it})$, i.e., the variance in the number of cellular transcript molecules between wells/cells.
However, this not possible as we are confounded by the unknown efficiencies $\delta_{it}$. 
Thus, we instead consider the log-library size $L_i$ for well $i$, which has the conditional distribution
\[
L_i | V_{ix}, V_{iy}, \psi_i, \nu_{i1}, \ldots, \nu_{in} \sim \mathcal{N}(\log ( \rho_i \textstyle\sum_t \delta_{it} \nu_{it}), \sigma^2_{lib}) \;.
\]
Again, $\sigma^2_{lib}$ represents variability due to sequencing.
We assume that the conditional distribution of $L_i$ is independent of the conditional distributions for $y_{ix}$ and $y_{iy}$, i.e., sampling during sequencing is independent.
We define the normalized log-library size by subtracting the log-total of the spike-in counts, i.e., $\tilde{L_i} = L_i - \log y_{ix}$.
Here, we have chosen population $x$, though $y$ can also be used.
$\tilde{L_i}$ has the conditional distribution
\[
\tilde{L_i} | V_{ix}, V_{iy}, \psi_i, \nu_{i1}, \ldots, \nu_{in} \sim \mathcal{N}( \log S_{ix} + \log(\textstyle\sum_t \delta_t \pi_{tx}) - \log(\textstyle\sum_t \delta_t \nu_{it}), 2\sigma^2_{lib})
\]
which no longer depends on $V_{iy}$ or $\psi_i$.
Now, recall that $\log V_{ix}$ is normally distributed, such that 
\[
\tilde{L_i} | \nu_{i1}, \ldots, \nu_{in} \sim \mathcal{N}( E(\log V_{ix}) + \log C_{x} + \log(\textstyle\sum_t \delta_t \pi_{tx}) - \log(\textstyle\sum_t \delta_t \nu_{it}), 2\sigma^2_{lib} + \sigma^2_{vol}) \;.
\]
We further assume that $\log(\textstyle\sum_t \delta_t \nu_{it})$ is normally distributed with variance $\sigma^2_{bio}$.
This variance represents the biological variability in the expected number of reads, which can be used as a proxy for the variability in the number of transcript molecules.
Indeed, one could argue that the former is more relevant than the latter --  sequencing returns results in terms of reads, not transcripts, and even UMI-based methods fail to capture all transcript molecules.
With the above assumption, $\tilde{L_i}$ becomes normally distributed with variance
\[
\mbox{var}(\tilde{L_i}) = 2\sigma^2_{lib} + \sigma^2_{vol} + \sigma^2_{bio} \;.
\]
Statistical methods can then be applied to estimate the value of $\sigma^2_{bio}$ from the sample variance of $\tilde{L_i}$ across cells/wells.
This can be compared to $\sigma^2_{vol}$ to determine the relative importance of spike-in variability.

% Note that under the null, i.e., vol=bio, the variance of L_i should be equal to that of R_i.
% However, testing is complicated by the fact that we're using population 'x' twice in each well, to compute L_i and R_i.
% We'd end up with some correlation in the variance estimates, which would be pretty annoying.
% This isn't easily resolved; L_i and R_i are normally distributed, but y_ix and company are not (e.g., the denominator of rho_i consists of sum of log-normals, at best).
% I guess we could get independence by splitting the wells into two groups, computing L_i from one group and R_i from the other, and using that for testing. 
% We can probably afford to do this if we have enough wells to play around with (96 per plate?).

\end{document}


