\documentclass{article}
\usepackage{graphicx}
\usepackage[margin=2.5cm]{geometry}
\usepackage[labelfont=bf]{caption}
\usepackage{color}
\usepackage{xcite}
\usepackage{amsmath}
\usepackage{textcomp}
\usepackage{natbib}

\renewcommand{\textfraction}{1.0}
\renewcommand{\floatpagefraction}{.9}
\newcommand\revised[1]{\textcolor{red}{#1}}
\renewcommand{\topfraction}{0.9}    % max fraction of floats at top
\renewcommand{\bottomfraction}{0.8} % max fraction of floats at bottom
\renewcommand{\textfraction}{0.07}  % allow minimal text w. figs

\makeatletter 
\renewcommand{\thefigure}{S\@arabic\c@figure} 
\renewcommand{\thetable}{S\@arabic\c@table} 

\usepackage{url}
\urlstyle{same}

\begin{document}

\begin{titlepage}
\vspace*{3cm}
\begin{center}

{\LARGE
When normalization gets prickly: are spike-ins good enough for single-cell RNA sequencing data?
\par}

\vspace{0.75cm}

{\Large 
    \textsc{Supplementary Materials}
\par
}
\vspace{0.75cm}

\large
by


\vspace{0.75cm}
Aaron T. L. Lun$^{1}$ and John C. Marioni$^{1,2}$

\vspace{1cm}
\begin{minipage}{0.9\textwidth}
\begin{flushleft} 
$^1$Cancer Research UK Cambridge Institute, University of Cambridge, Li Ka Shing Centre, Robinson Way, Cambridge CB2 0RE, United Kingdom \\[6pt]
$^2$EMBL European Bioinformatics Institute, Wellcome Genome Campus, Hinxton, Cambridge CB10 1SD, United Kingdom \\[6pt]
\end{flushleft}
\end{minipage}

\vspace{1.5cm}
{\large \today{}}

\vspace*{\fill}
\end{center}
\end{titlepage}

\section{Introducing the theoretical framework}
For each well $i$, we consider the following notation:
\begin{center}
\begin{tabular}{l l}
$C_{s}$ & The concentration in transcript molecules per \textmu{}L for spike-in population $s$; a constant. \\
$V_{is}$ & The volume of spike-in population added to a well; a random variable. 
\end{tabular}
\end{center}
We define $S_{is} = C_{s}V_{is}$, the number of transcript molecules for population $s$ added to $i$ during spike-in addition.
Further consider the following definitions:
\begin{center}
\begin{tabular}{l l}
$\pi_{ts}$ & The proportion of molecules for transcript $t$ in spike-in population $s$; a constant. \\
$\delta_{t}$ & The optimal number of cDNA fragments generated from each molecule of transcript $t$; a constant. \\ 
$\psi_i$ & The efficiency of cDNA generation in well $i$; a random variable.
\end{tabular}
\end{center}
We further define the random variable $\delta_{it} = \psi_i\delta_t$, which represents the cDNA fragment-per-transcript molecule conversion rate for transcript $t$ in well $i$.
Finally, define $\nu_{it}$ as the number of molecules for transcript $t$ in the cellular RNA for the cell in well $i$ -- this is another random variable across wells.

All random variables are assumed to be independent unless they have been explicitly related by equalities, e.g., $S_{is}$ and $V_{is}$.
This is justified by the fact that they describe separate processes in the protocol.

For convenience, we will consider the union of all transcripts when referring to the sum across all $t$.
This includes both spike-in and cellular transcipts, with a total of $n$ distinct species.
Obviously, this means that $\pi_{ts}=0$ for transcripts that only exist in the cell, while $\nu_{it}=0$ for spike-in-only transcripts. \\[0.1in]

For the first experiment, assume that two spike-in populations are present, i.e., $s=x,y$.
The expected total read count across all genes in $s$ is affected by library size and undersampling, and is defined as
\[
\mu_{is} = \left[\frac{N_i}{S_{ix}\sum_t \delta_{it}\pi_{tx} + S_{iy}\sum_t \delta_{it}\pi_{ty} + \sum_t \delta_{it} \nu_{it}}\right] S_{is}\sum_t \delta_{it}\pi_{ts}
= \rho_i  S_{is}\sum_t \delta_{it}\pi_{ts}
\]
where $N_i$ is the expected total number of (mapped) reads in the library generated from well $i$.
Note that $\rho_i=1$ when sequencing is performed to saturation, e.g., for UMIs -- this will not affect the generality of our conclusions below, as $\rho_i$ will cancel out in calculation of the relevant (log-)ratios.
The observed total read count $y_{is}$ is assumed to be log-normally distributed, conditional on the other random variables, i.e., 
\[
\log y_{is} | V_{ix}, V_{iy}, \psi_i, \nu_{i1}, \ldots, \nu_{in} \sim \mathcal{N}(\mu_{is}, \sigma^2_{lib})
\]
where $\sigma^2_{lib}$ represents the variability due to sequencing.
Counts from each population are independent.
The log-ratio of the total counts between populations is $R_i = \log(y_{ix}/y_{iy})$, and is conditionally distributed as
\[
R_i | V_{ix}, V_{iy},  \psi_i, \nu_{i1}, \ldots, \nu_{in} \sim \mathcal{N}( \log \mu_{ix} - \log \mu_{iy}, 2\sigma^2_{lib} ) \;. 
\]
There is actually no dependence on $\psi_i$ or $\nu_{it}$, as these terms cancel out during calculation of $R_i$.
To illustrate, 
\begin{align*}
E(R_i | V_{ix}, V_{iy},  \psi_i, \nu_{i1}, \ldots, \nu_{in} ) &= \log \mu_{ix} - \log \mu_{iy} \\
&= \log \frac{ S_{ix } \sum_t \delta_{it} \pi_{tx} } { S_{iy}\sum_t \delta_{it} \pi_{ty} } \\
&= \log V_{ix} - \log V_{iy} + \log (C_x/C_y) + \log (\textstyle\sum_t \delta_{t} \pi_{tx}/\textstyle\sum_t \delta_{t} \pi_{ty}) \\
&= \log V_{ix} - \log V_{iy } + Z \quad\mbox{for some constant } Z
\end{align*}
This allows us to simplify the conditional distribution from $R_i | V_{ix}, V_{iy},  \psi_i, \nu_{i1}, \ldots, \nu_{in}$ to $R_i | V_{ix}, V_{iy}$.
In particular, cancellation of $\psi_i$ allows us to ignore well-specific effects like RT efficiency or barcode biases.

Further assume that both $\log V_{ix}$ and $\log V_{iy}$ are normally distributed with variance $\sigma^2_{vol}$.
The use of the same variance is justified by the fact that the same spike-in volume is added for both $x$ and $y$, such that the variability should be the same for both volumes.
The assumption means that the conditional mean $E(R_i |V_{ix}, V_{iy})$ is normally distributed, such that $R_i$ itself is also normally distributed with variance
\[
\mbox{var}(R_i) = 2\sigma^2_{lib} + 2\sigma_{vol}^2 \;. 
% Easily derived using the law of total variances.
% Normal distribution for R_i is obvious, based on product of PDFs (int[ P(X=x|u)*P(u) du ] gives a exp(x^2), eventually, and you can work this out to validate the variance).
% Both claims tested in simulations (sample variances of normal distribution with normal mean match up with expected truth, no rejections on a Shapiro-Wilk test).
\]
This can be estimated as the sample variance of the log-ratios across all wells in the first experiment. \\[0.1in]

In the second experiment, mixing of spike-in populations occurs beforehand and is not variable.
More specifically, $V_{ix}=V_{iy}$ such that they cancel out in calculating $R_i$ and do not contribute to the variance.
This means that the variance of the log-ratio (denoted here as $R_i'$) across multiple runs can be described as
\[
\mbox{var}(R_i') = 2\sigma^2_{lib}
\]
and can be estimated as the sample variance of the log-ratios in this experiment.
One can see that the variance estimates for $\log R_i$ and $\log R_i'$ in the first and second experiments can be used to obtain an estimate of $\sigma^2_{vol}$ - most obviously via subtraction, though a more refined approach will be described below. 

At this point, it is worth taking a closer look at the assumption used to derive $\sigma^2_{lib}$.
We have assumed that this value is constant with respect to the random variables $V_{is}$, $\psi_i$ and $\nu_{it}$.
However, this may not be true as the variance of count data will vary according to the mean.
One would then expect that the variance of the log-ratio of counts would depend on the magnitude of $V_{ix}$, $V_{iy}$ and others.
Accounting for this effect would be problematic, as it would complicate the expressions for the variances.
In practice, we should be able to ignore this dependency as long as the counts are large enough for both spike-in populations.
This can be done by carefully choosing the concentrations of the spike-in RNA so that enough RNA is sequenced from each population.
The sampling variability of $\log y_{is}$ should approach zero at large means if counts are Poisson-distributed, or approach a constant under the NB distribution with a fixed dispersion.
The former can be validated by checking whether $\mbox{var}(R_i')$ is much smaller than $\mbox{var}(R_i)$.
While this may not be the case if both are near zero, any inaccuracy becomes irrelevant as we can simply state that spike-in addition is precise.
The latter is justified by the plateau in the NB dispersions at large means in most real data sets.
In both cases, any change with respect to the random variables can be considered to be negligible. \\[0.1in]

% It's possible to do something more complicated with a Poisson sampling distribution that's conditional on the mean.
% You could use the general variance decomposition formula to get the variance of the total count (see Bowsher and Swain, 2012).
% This would account for the mean-variance relationship at low counts.
% However, it becomes intractable to solve for the variance of the log-ratio of counts.
% There is no apparent benefit to this complexity, given that the log-normal approximation is more than sufficient at large counts.

\newpage
\section{Dealing with variable spike-in behaviour}
Another implicit assumption is that the spike-in populations exhibit equivalent behaviour with respect to cDNA generation efficiency.
This allows $\psi_i$ to cancel out during calculation of $R_i$, such that any variability in efficiency across wells can be ignored.
However, this may not be true for populations like the set of ERCC spike-ins with, e.g., differing GC composition, shorter poly-A tails.
Differences in well-specific efficiency can be modelled by introducing a $\psi_{is}$ term to replace $\psi_i$ in the above calculations.
This means that the conditional expectation of $R_i$ with respect to the random variables $V_{i}$ and $\psi_{is}$ becomes
\[
E(R_i | V_{ix}, V_{iy},  \psi_{ix}, \psi_{iy} ) = \log \frac{ S_{ix} \sum_t \psi_{ix}\delta_{t} \pi_{tx} } { S_{iy}\sum_t \psi_{iy}\delta_{t} \pi_{ty} } \;. \\
\]
The effect on the variance of $R_i$ depends on whether the ratio $\psi_{ix}/\psi_{iy}$ is constant or variable.
If it is constant, it will be absorbed into the constant term $Z$ and will not contribute to the variance across wells.
Otherwise, we can model $\log(\psi_{ix}/\psi_{iy})$ with a normal distribution that is independent of $V_{is}$.
The variance of this distribution will contribute directly to both $\mbox{var}(R_i)$ and $\mbox{var}(R_i')$, and will be effectively calculated as part of $\sigma^2_{lib}$.
Subtraction of sample variances will subsequently yield an estimate of $\sigma^2_{vol}$, as before.

Disentangling the variability of $\log(\psi_{ix}/\psi_{iy})$ from the other components of $\sigma^2_{lib}$ is more difficult.
It requires careful control over the properties of the spike-in populations -- namely, by using two populations that are effectively identical for the purposes of reverse transcription, yet sufficiently different for read alignment to the correct genome.
Some study of the relative contribution of this variance term can be performed by using several different spike-ins, and observing differences in the estimated $\sigma^2_{lib}$ (assuming that variance in the other steps of the protocol are constant across runs).
If the variance is negligible, then the exact value of $\psi_{ix}/\psi_{iy}$ is irrelevant in most applications as it will have no effect on the normalization factors between cells.

Another approach is to shuffle the spike-in sets \textit{in silico} for the second experiment.
Half the species in one set are summed with half the species in the other set, and this is repeated with the remaining species in both sets.
The variance of the log-ratios between these two sums is computed across wells, using the same partitioning of species for all wells.
To ensure robustness, a different partition can be chosen and the variance calculation repeated; the average variance can be computed across partitions.
The idea is that, on average, the shuffled populations do not exhibit any systematic difference in their capture efficiency.
Any variability in the log-ratio is attributable to the protocol alone, rather than changes to efficiency.
Subtracting from the original variance of the control experiment will identify the variability due to spike-in behaviour.

\newpage
\section{Statistical analyses of the results}
Denote the sample variance of $R$ as $s^2_1$, and that of $R'$ as $s^2_2$.
These are estimated from the set of observed ratios in each experiment, using the corrected Pearson estimator.
We could then intuitively define
\begin{align*}
\widehat{\sigma^2_{lib}} &= s^2_2 \quad\mbox{and} \\
\widehat{\sigma^2_{vol}} &= \frac{s^2_1 - s^2_2}{2} \;.
\end{align*}
However, this is not sensible if $s^2_1 < s^2_2$ as the estimate for $\sigma^2_{vol}$ will be negative.
In such cases, 
\begin{align*}
\widehat{\sigma^2_{vol}} &= 0 \quad\mbox{and} \\
\widehat{\sigma^2_{lib}} &= \frac{s^2_1(n_1 - 1) + s^2_2(n_2-1)}{n_1 + n_2 - 2} 
\end{align*}
where $n_1$ and $n_2$ are the number of repeated wells in the first and second experiments, respectively.
This uses all of the data from the first and second experiments to estimate $\sigma^2_{lib}$, while the estimate of $\sigma^2_{vol}$ is fixed at the lower bound of zero.
These definitions are based on the principle of restricted maximum likelihood when the parameter space is constrained to non-negative values (see Thompson, 1962).

% See "The Problem of Negative Estimates of Variance Components". Thompson (1962), Ann. Math. Statist., Volume 33, Number 1 (1962), 273-289.
% More specifically, the log-REML is equal to:
%\[
%L(\mathbf{y_1}, \mathbf{y_2} | \sigma^2_{lib}, \sigma^2_{vol}) = - n_1 \log(\sigma^2_{lib}) - \frac{1}{\sigma^2_{lib}} \sum y_{1i} - n_2 \log(\sigma^2_{lib} + \sigma^2_{vol}) - \frac{1}{\sigma^2_{lib} + \sigma^2_{vol}} \sum y_{2i} 
%\]
% where y_1 and y_2 are vectors of residual effects from the two experiments.
% If you take the partial derivatives with respect to each sigma^2, you get two equations that can be solved simultaneously.
% This yields a single maximum for the REML at the standard expressions for the experimental sample variances (and after subtraction, for the volume variance).
% However, the story's different when the volume variance estimate is forced to zero.
% Under normal conditions, there's only one maxima (see above) - but in this case, the maxima lies outside the constrained space.
% This means that the maxima in the constrained space must lie on the boundary, as the REML must increase as it gets closer to the maxima.
% We can then plug in a zero value for the volume variance and solve for the library variance, which gives us the value above. 

It is also possible to determine the significance of non-zero estimates for $\sigma^2_{vol}$.
This aims to determine whether there is a significant increase in variability from volume addition, over the variability introduced during general library preparation.
Specifically, the null hypothesis states that $\sigma^2_{vol} = 0$.
This means that
\begin{align*}
s^2_1 &\sim \frac{\sigma^2_{lib} \chi^2_{n_1 - 1}}{n_1 - 1} \quad\mbox{and} \\
s^2_2 &\sim \frac{\sigma^2_{lib} \chi^2_{n_2 - 1}}{n_2 - 1} 
\end{align*}
As the sample variances are estimated independently, the ratio $s^2_1/s^2_2$ follows a F-distribution on $n_1-1$ and $n_2-1$ degrees of freedom under the null.
The observed value of this ratio can be used to compute a $p$-value based on the upper tail of this distribution.
A similar test can be used to determine whether alterations in the protocol result in significant decreases in the variance, i.e., improvements in spike-in reliability.

Of course, these analyses hinge on the assumption of normality for the log-ratios. 
This can be evaluated empirically using established statistical methods such as the Shapiro-Wilk test.
Simulations indicate that the log-ratios of overdispersed count data can be accurately modelled with the normal distribution.

An alternative approach to variance estimation is to take the median absolute deviation (MAD) for each set of ratios and multiply it by 1.4826.
This will yield a robust, unbiased estimate of the standard deviation under normality.
While this protects against outliers, it will likely invalidate the tests described above -- the variance of this estimator will be different from the Pearson estimator.
Rather, it may be possible to use simulations to obtain a null distribution for the variance ratios.
For example, you could simulate the ratios of MAD-based variance estimates for the standard normal distribution.
The final distribution of these ratios should be the same even if you change the location or scale of the original normal distribution.

\newpage
\section{Estimating the variability of total cellular RNA}
Our aim is to estimate the variance of total cellular RNA, and to compare it to the variance of the spike-ins.
If the former is negligible compared to the latter, we can claim that any imprecision in spike-in addition will have little impact on the biological conclusions.
This suggests that spike-ins can be safely used.

To test this, we would ideally estimate $\mbox{var}(\sum_t \nu_{it})$, i.e., the variance in the number of cellular transcript molecules between wells/cells.
However, this not possible as we are confounded by the unknown efficiencies $\delta_{it}$. 
Thus, we instead consider the log-library size $L_i$ for well $i$, which has the conditional distribution
\[
L_i | V_{ix}, V_{iy}, \psi_i, \nu_{i1}, \ldots, \nu_{in} \sim \mathcal{N}(\log ( \rho_i \textstyle\sum_t \delta_{it} \nu_{it}), \sigma^2_{lib}) \;.
\]
Again, $\sigma^2_{lib}$ represents variability due to sequencing.
We assume that the conditional distribution of $L_i$ is independent of the conditional distributions for $y_{ix}$ and $y_{iy}$, i.e., sampling during sequencing is independent.
We define the normalized log-library size by subtracting the log-total of the spike-in counts, i.e., $\tilde{L_i} = L_i - \log y_{ix}$.
Here, we have chosen population $x$, though $y$ can also be used.
$\tilde{L_i}$ has the conditional distribution
\[
\tilde{L_i} | V_{ix}, V_{iy}, \psi_i, \nu_{i1}, \ldots, \nu_{in} \sim \mathcal{N}( \log S_{ix} + \log(\textstyle\sum_t \delta_t \pi_{tx}) - \log(\textstyle\sum_t \delta_t \nu_{it}), 2\sigma^2_{lib})
\]
which no longer depends on $V_{iy}$ or $\psi_i$.
Now, recall that $\log V_{ix}$ is normally distributed, such that 
\[
\tilde{L_i} | \nu_{i1}, \ldots, \nu_{in} \sim \mathcal{N}( E(\log V_{ix}) + \log C_{x} + \log(\textstyle\sum_t \delta_t \pi_{tx}) - \log(\textstyle\sum_t \delta_t \nu_{it}), 2\sigma^2_{lib} + \sigma^2_{vol}) \;.
\]
We further assume that $\log(\textstyle\sum_t \delta_t \nu_{it})$ is normally distributed with variance $\sigma^2_{bio}$.
This variance represents the biological variability in the expected number of reads, which can be used as a proxy for the variability in the number of transcript molecules.
Indeed, one could argue that the former is more relevant than the latter --  sequencing returns results in terms of reads, not transcripts, and even UMI-based methods fail to capture all transcript molecules.
With the above assumption, $\tilde{L_i}$ becomes normally distributed with variance
\[
\mbox{var}(\tilde{L_i}) = 2\sigma^2_{lib} + \sigma^2_{vol} + \sigma^2_{bio} \;.
\]
Statistical methods can then be applied to estimate the value of $\sigma^2_{bio}$ from the sample variance of $\tilde{L_i}$ across cells/wells.
This can be compared to $\sigma^2_{vol}$ to determine the relative importance of spike-in variability.

% Note that under the null, i.e., vol=bio, the variance of L_i should be equal to that of R_i.
% However, testing is complicated by the fact that we're using population 'x' twice in each well, to compute L_i and R_i.
% We'd end up with some correlation in the variance estimates, which would be pretty annoying.
% This isn't easily resolved; L_i and R_i are normally distributed, but y_ix and company are not (e.g., the denominator of rho_i consists of sum of log-normals, at best).
% I guess we could get independence by splitting the wells into two groups, computing L_i from one group and R_i from the other, and using that for testing. 
% We can probably afford to do this if we have enough wells to play around with (96 per plate?).

\end{document}


